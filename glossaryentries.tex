\newcommand{\FullGlossaryEntry}[7]{%Abkz., Typ, Name, Name(plural), Beschr., Beschr.(plural), Erkl.
\newglossaryentry{#1}{type=\acronymtype,
	name={#3},
	plural={#4},
	description={\textit{#5}},
	%first={\textit{#5}(later referred to as #3)\glsadd{_#1}},
	%firstplural={\textit{#6}(later referred to as #4)\glsadd{_#1}},
	first={#5(#3)\glsadd{_#1}},
	firstplural={#6(#4)\glsadd{_#1}},
	see=[See]{_#1}
}
\longnewglossaryentry{_#1}{type=#2,
	name={#5},
	plural={#6}
	}{\hspace{0pt}\\#7}
}


% Acronyms
%\newacronym{jlu}{JLU}{Justus-Liebig-Universität}
%\newacronym{hrz}{HRZ}{Hochschulrechenzentrum}
%\newacronym[plural=LEDs, longplural={light-emitting diodes}]{led}{LED}{light-emitting diode}
%\newacronym[plural=EEPROMs, longplural={electrically erasable programmable read-only memories}]{eeprom}{EEPROM}{electrically erasable programmable read-only memory}

%  Glossary Entries
\newglossaryentry{port}{
	name=port,
	description={A point for traffic to flow, represented by an unsigned integer of up to 2 bytes. The name was chosen as an analogy to ports for ships.},
	plural=ports
}
\newglossaryentry{shell}{
	name=shell,
	description={A \gls{CLI} program, that reads user input line-by-line and executes those commands.},
	plural=shells
}
\newglossaryentry{pipe}{
	name=pipe,
	description={A pipe is a \gls{Unix} feature for \gls{IPC}.},
	plural=pipes
}
\newglossaryentry{socket}{
	name=socket,
	description={A network socket is an endpoint for communication over \glspl{port}.},
	plural=sockets
}
\newglossaryentry{terminal}{
	name=terminal,
	description={An user interface to interact with \gls{CLI} programs like \glspl{shell}.},
	plural=terminals
}
\newglossaryentry{Unix}{
	name=Unix,
	description={An originally free \gls{OS} family called Unics from AT\&T that re-imagined an older \gls{OS} by the name of Multics.}
}
\newglossaryentry{Linux}{
	name=Linux,
	description={A \gls{Unix} based \gls{OS} which uses Linus Torvalds kernel and was inspired by Minix.}
}
\newglossaryentry{Request Broker}{
	name=Request Broker,
	description={A service that oversees the action requests of a program and decides whether to permit and execute them or not, based on various factors.},
	plural=Request Brockers
}

\newglossaryentry{CGo}{
	name=CGo,
	description={\gls{C} support for \gls{Go}.}
}
\newglossaryentry{x-package}{
	name=x-package,
	description={A \gls{Go} package that is not part of the standard library and that is subject to change or even entirely disappear.}
	plural=x-packages
}
\newglossaryentry{X.509}{
	name=X.509,
	description={An international standard for certificates in public key infrastructures.}
}
\newglossaryentry{DiffieHellmanKX}{
	name={Diffie-Hellman key exchange},
	description={A key exchange algorithm that prevents exposure to eavesdropping third parties.},
	plural={Diffie-Hellman key exchanges}
}

% Acronyms
\newglossaryentry{OS}{type=\acronymtype,
name={OS},
plural={OSes},
description={\textit{Operating System}},
first={Operating System(OS)},
firstplural={Operating Systems(OSes)}
}

% Acronyms with Glossary Entries
\FullGlossaryEntry{ZHAW}{main}%
{ZHAW}{}%
{Zurich University of Applied Sciences}{}%
{Name of my university of trust.}

\FullGlossaryEntry{Go}{main}%
{Go}{}%
{the Go/Golang programming language}{}%
{Google's programming language.}

\FullGlossaryEntry{C}{main}%
{C}{}%
{the C programming language}{}%
{Low-level programming language originally invented by Dennis Ritchie.}

\FullGlossaryEntry{C++}{main}%
{C++}{}%
{the C++ programming language}{}%
{Descendant of \gls{C} which implemented \gls{OOP}.}

\FullGlossaryEntry{OOP}{main}%
{OOP}{}%
{Object Oriented Programming}{}%
{A programming paradigm which uses objects to model real life entities.}

\FullGlossaryEntry{fd}{main}%
{fd}{fds}%
{file descriptor}{file descriptors}%
{A file descriptor is an integer that represents the handle to a file.}

\FullGlossaryEntry{CLI}{main}%
{CLI}{}%
{Command Line Interface}{}%
{A text based interface centered around commands to perform specific tasks.}

\FullGlossaryEntry{tty}{main}%
{tty}{ttys}%
{teletype}{teletypes}%
{Originally a device that could send and receive text messages. Nowadays \gls{tty} refers to \glspl{terminal}, which emulate that behaviour\citep{tty}.}

\FullGlossaryEntry{pty}{main}%
{pty}{ptys}%
{pseudoterminal}{pseudoterminals}%
{A mechanism of \gls{Unix} to allow programs to communicate as if the other was inside a \gls{tty}\citep{pty}.}

\FullGlossaryEntry{ptm}{main}%
{ptm}{ptms}%
{pseudoterminal master}{pseudoterminal masters}%
{The master of a \gls{pty}.}

\FullGlossaryEntry{pts}{main}%
{pts}{pts's}%
{pseudoterminal slave}{pseudoterminal slaves}%
{The slave of a \gls{pty}\citep{pts}.}

\FullGlossaryEntry{SIGINT}{main}%
{SIGINT}{SIGINTs}%
{interrupt signal}{interrupt signals}%
{A signal supported by \gls{Unix} based \glspl{OS} which signals to a process to interrupt it's work.}

\FullGlossaryEntry{SSH}{main}%
{SSH}{}%
{Secure Shell}{}%
{An client-server-application that allows remote login and interaction with a \gls{shell}. See \ref{ssec:OpenSSH}.}

\FullGlossaryEntry{PAM}{main}%
{PAM}{PAMs}%
{Pluggable Authentication Module}{Pluggable Authentication Modules}%
{Modules for user authentication.}

\FullGlossaryEntry{API}{main}%
{API}{APIs}%
{Application Programming Interface}{Application Programming Interfaces}%
{Accessible interface for developers to use external code.}

\FullGlossaryEntry{PID}{main}%
{PID}{PIDs}%
{Process ID}{Process IDs}%
{The unique identifier of a process represented as an integer.}

\FullGlossaryEntry{IPC}{main}%
{IPC}{}%
{Inter-Process-Communication}{}%
{Communication between processes.}

\FullGlossaryEntry{GUI}{main}%
{GUI}{GUIs}%
{Graphical User Interface}{Graphical User Interfaces}%
{Graphical interface for the user to visually interact with a program.}

\FullGlossaryEntry{SSL}{main}%
{SSL}{}%
{Secure Sockets Layer}{}%
{Cryptographic protocol to secure the communication between two peers via symmetric cryptography. Deprecated.}

\FullGlossaryEntry{TLS}{main}%
{TLS}{}%
{Transport Layer Security}{}%
{Newer and recommended version of \gls{SSL}.}

\FullGlossaryEntry{TELNETS}{main}%
{TELNETS}{}%
{Telnet Secure}{}%
{Telnet with \gls{SSL} encryption.}


% Symbols
%\newglossaryentry{ohm}{type=symbols, name={\ensuremath{\Omega}}, sort=Ohm, symbol={\ensuremath{\Omega}}, description={unit of electrical resistance}}
%\newglossaryentry{angstrom}{type=symbols, name={\AA}, sort=angström, symbol={\AA}, description={non-SI unit of length}}